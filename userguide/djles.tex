\documentclass[letterpaper]{article}
\usepackage{amsmath} 
\usepackage{fullpage}
\usepackage{hyperref}
\usepackage{natbib}
\usepackage{textcomp}

\newcommand{\deriv}[2]{\frac{\mathrm{d}#1}{\mathrm{d}#2}}

\begin{document}
\newpage
\pagestyle{plain}
\setcounter{page}{1}

\begin{center}
\huge{DJLES: Dubreil-Jacotin-Long Equation Solver}\\[1em]
\large{Nov 02, 2015}\\
Michael Dunphy
\end{center}

\section{Introduction}
DJLES is a MATLAB package that finds a mode-one solution to the DJL equation. \cite{StastnaLamb2002} describe the problem setup, weakly nonlinear theory, fully nonlinear theory, and iterative solution procedure. DJLES was inspired by \verb+soliw+, a DJL solving C package resulting from the work of \cite{StastnaLamb2002}.

DJLES differs from \verb+soliw+ by being MATLAB based, which avoids the need to compile C code and import the resulting raw binary files into MATLAB. It also takes advantage of MATLAB's built in parallelisation of FFTs which accelerates solution on multi-core computers. Lastly, it easily employs the resolution refinement approach of \cite{DunphySubichStastna2011}, a feature not available in \verb+soliw+

If you find this code useful, please cite \cite{DunphySubichStastna2011} as it will encourage the author to maintain/improve DJLES. Bug reports, fixes/improvements and additional case files are welcome; please send them to \href{mailto:mdunphy@uwaterloo.ca}{mdunphy@uwaterloo.ca} for inclusion in a future version.

DLJES has benefited from contributions from Marek Stastna, Kevin Lamb, Chris Subich and Derek Steinmoeller.

The remainder of this document is a user guide for the code.

\section{Usage}
Typical usage is to set problem parameters, then call \verb+djles_refine_solution+ to solve the DJL equation, which returns the isopycnal displacement field $\eta$ and wave speed $c$ in MATLAB variables \verb+eta+ and \verb+c+, respectively. The ten required problem parameters are:
\begin{itemize}
 \item $A$ - Target available potential energy (m$^4$/s$^2$)
 \item $L$ and $H$ - Domain length and depth (m)
 \item $N_x$ and $N_z$ - Number of grid points in $x$ and $z$
 \item $\bar{\rho}(z)$ and $\deriv{}{z} \bar{\rho}(z)$
       - Background density profile (unitless: normalized by $\rho_0$) and first derivative
  \item $U_{bg}(z)$, $\deriv{}{z} U_{bg}(z)$, and $\deriv{^2}{z^2} U_{bg}(z)$
       - Background velocity profile (m/s) and first two derivatives
\end{itemize}
If a background velocity is not desired, set it to zero (see test cases). 
Once the solution is complete, \verb+djles_diagnostics+ computes the associated velocity fields, vorticity, etc., and \verb+djles_plot+ produces a simple plot of these fields.

During the first call to \verb+djles_refine_solution+, we use weakly nonlinear theory to find an initial guess for the solution, which is then iterated to convergence. Subsequent calls to \verb+djles_refine_solution+ use the previous fully nonlinear solution as the initial guess. This allows for the strategy of successively adjusting parameters (resolution, wave APE, etc) to ease/accelerate finding a solution. The various test cases demonstrate several solution scenarios.

\subsection{Tuning knobs}
There are a few tuning knobs. Set them in the case file to override the default.
\begin{itemize}
 \item \verb+min_iterate+ specifies the minimum number of iterations that the solver will take, default is 10.
 \item \verb+max_iterate+ specifies the maximum number of iterations that the solver will take, default is 2000. If you use a very small \verb+epsilon+ or \verb+relax+ you may need to increase this. 
 \item \verb+NL+ is the number of Legendre points to use for the Gauss quadrature used for the APE integral. Values below 5 are probably too small, values in 15-25 are quite good, and values higher than 25 are likely unnecessary. Default is 20.
 \item \verb+relax+ is the underrelaxation factor, $0 < \verb+relax+ <= 1$.
        The value is the fraction of the new iteration, $\eta^{k*}$ to retain
        after performing an iteration, that is,
        $\eta^k = \eta^{k-1} + \verb+relax+(\eta^{k*} - \eta^{k-1})$.
        The default of 0.5 works well in most cases, and a value of 1 disables underrelaxation.
        Using a small value can help stabilise the solver at the expense of needing more iterations.
 \item \verb+g+ is the gravitational acceleration constant, default value is 9.81 m/s$^2$.
 \item \verb+verbose+ is a switch that enables printing the progress as the iterations proceed. Setting to 0 disables printing, and to 1 enabled printing. Default is 1.
\item \verb+epsilon+ controls the stop condition, which is the relative difference between successive iterations. Default $1\times10^{-4}$. Smaller values will require more solver iterations.
\end{itemize}

\subsection{Miscellaneous}

A successful solve will have a relative error of roughly \verb+epsilon+/\verb+relax+ decimal places in $\eta$. Using a small value of \verb+relax+ should be accompanied with a reduction in \verb+epsilon+.

If the solver fails to converge, 
\begin{enumerate}
 \item Ensure that your density profile $\rho(z)$ is stably stratified.
 \item Run the solver with a small underrelaxation factor. A value of 0.1 or even 0.01 may suppress oscillations and stabilise the solver. 
 \item Ensure that your density profile (and stratification profile) is smooth.
\end{enumerate}

\subsection{Compatibility notes}
\begin{itemize}
 \item Works with MATLAB versions: R2013a (Linux 64-bit and Windows), R2012a (OSX 10.6.8)
 \item Works with Octave v3.8.1 (Linux 64-bit).
 \item The initial guess part uses \verb+polyeig.m+, and Octave added it in v3.8.0, so it's unlikely to work with Octave older than v3.8.0.
 \item MATLAB R2014a prints warnings about deprecating \verb+ppval+, you can suppress the warnings with \\ \verb+warning('off','MATLAB:interp1:ppGriddedInterpolant');+
\end{itemize}

\subsection{Numerical Grid}
The domain is the rectangular region $[0,L] \times [-H,0]$ and is discretized into $N_x\times N_z$ grid boxes. The boundary conditions for the problem are zero, so we use the cell centres for the coordinates, that is,
\begin{align}
x_i^c &= \frac{\Delta x}{2}  + i \Delta x, \quad i \in 0,1,2,\hdots,N_x-1,\\
z_j^c &= -H + \frac{\Delta z}{2}  + j \Delta z, \quad j  \in 0,1,2,\hdots,N_z-1,
\end{align}
where $(\Delta x, \Delta z) = (L/N_x, H/N_z)$. The grid shifting routine \verb+djles_shift_grid.m+ shifts the data from the cell centres to cell edges, 
\begin{align}
x_i^e &= i \Delta x, \quad i \in 0,1,2,...,N_x\\
z_j^e &= -H + j \Delta z, \quad j  \in 0,1,2,...,N_z.
\end{align}
The array sizes are $N_x+1 \times N_z+1$ on the cell edges grid.

\section{Description of M-files}
The code consists of a handful of functions, scripts, and tests cases.

\subsection{Scripts}
The scripts are
\begin{verbatim}
Filename                  - Description
-----------------------------------------------------------------------------------------
djles_common.m            - Sets default parameters and generates grid and wavenumbers.
djles_diagnostics.m       - Computes wave velocities, vorticity, etc from eta and c
djles_initial_guess.m     - Used by djles_refine_solution.m to get an initial guess for
                            eta & c from WNL theory if an initial guess is not provided.
djles_plot.m              - Plots the solution and some diagnostics
djles_refine_solution.m   - This is the main m-file that each case files calls. It
                            performs the iterative procedure to find a wave solution.
\end{verbatim}

\subsection{Helper Functions}
The helper functions are
\begin{verbatim}
djles_compute_apedens.m   - Computes the available potential energy density.
djles_diffmatrix.m        - Generates a finite difference differentiation matrix for use
                            in differentiating data.
djles_gradient.m          - Computes the gradient of a scalar function.
djles_residual.m          - Computes the residual in the DJL equation.
djles_raise_resolution.m  - Increases the resolution of the solution.
djles_shift_grid.m        - Shifts the eta solution from cell-centred grid to a cell
                            edge grid. Increases the matrix size by 1 in each dimension.
djles_sinequadrature.m    - Computed the interior grid sine quadrature weights for
                            area integrals.
\end{verbatim}

\subsection{Test cases}
Included are six test cases that demonstrate how to use DJLES. A brief description of each follows.

\subsubsection{Small APE}
Test case: \verb"case_small_ape.m" \\\\
This test case demonstrates the parameter regime of no background current, a smooth pycnocline, and small wave APE.
The initial guess obtained from weakly nonlinear theory will be ``close'' to the fully nonlinear DJL solution, so the solver readily converges.
Once we find the low resolution wave (32x32), we increase the resolution to 512x256 and iterate to convergence.
The low resolution solution is a very good initial guess for the high resolution problem, so the high resolution wave is also readily solved.

\subsubsection{Large APE}
Test case: \verb"case_large_ape.m" \\\\
This test case demonstrates the parameter regime of no background current, a smooth pycnocline, and large wave APE.
The initial guess from weakly nonlinear theory will be rather poor. 
The solution strategy we use is to find a wave with a small APE, and supply that wave as the initial guess to find a wave with a larger APE.
Applying this several times yields the final wave with a large APE.
In this case we start at one percent of the target APE and raise the APE in five steps.
Lastly, we increase the resolution once we reach the target APE.
The result is a broad flat crested wave.

\subsubsection{Background Current}
Test case: \verb"case_ubg.m" \\\\
The weakly nonlinear theory does not include a background current, and thus produces a mediocre initial guess.
However, the same successive solution strategy is useful here.
We solve without a background current first, and then successively raise the background current strength until we reach the full background current.
We increase epsilon for the intermediate solutions to accelerate the process as we only need them to be of ``initial guess'' precision.
Lastly, we reduce epsilon, and increase the resolution in two increments.

\subsubsection{Sharp Pycnocline}
Test case: \verb"case_sharp_pycnocline.m" \\\\
Obtaining a solution with a sharp pycnocline can be difficult for the solver to find directly, and the successive solution strategy works here as well.
We begin with a low-resolution wide-pycnocline that is readily solved and begin successively reducing the pycnocline thickness.
As we sharpen the pycnocline, we also increase the resolution such that the pycnocline is not under-resolved.

\subsubsection{Synthetic Data}
Test case: \verb"case_synthetic_data.m" \\\\
This case matches the parameters of the background current case, except that we use synthetic mooring data as the background profiles.
We sample the analytic functions at 25 evenly spaced points and construct continuous profiles using linear interpolating polynomials.
Second order differentiation provides the gradients, which are linearly interpolated in the same fashion.
The \verb+linear+ and \verb+pchip+ interpolation methods preserve monotonocity which is important to ensure the interpolated density profile is stable.
The \verb+spline+ method does not preserve monotonicity so be careful if you use it.
As in \verb"case_ubg.m", we raise the background velocity in four successive steps, followed by three resolution refinements.

\subsubsection{Real Data}
Test case: \verb"case_realdata.m" \&  \verb"case_realdata.txt" \\\\
Here we use time-averaged temperature profile data collected from Lake Erie (K. Lamb 2014, pers. comm.). 
The text file contains two columns, depth (m) and temperature (\textdegree C), and we convert the temperature to density with a linear equation of state.
There is no surface value so we extrapolate to find a value for $z=0$.
We construct the density and density gradient in the same fashion as in the synthetic data case, and use no background current.
We solve the wave at low resolution and then raise the resolution in two steps.

% References
\bibliographystyle{kbib}
\bibliography{djles}

\end{document}
